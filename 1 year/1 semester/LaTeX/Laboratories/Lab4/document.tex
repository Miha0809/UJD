\documentclass{article}
\usepackage[utf8]{inputenc}
\usepackage{color}
\usepackage[dvipsnames]{xcolor}
\usepackage{hyperref}
\usepackage[sharp]{easylist}
\usepackage[50]{easylist}
\author{Mykhailo Hulii}
\date{November 2022}

\begin{document}
	\newcounter{ListCounter}
	\newenvironment{Pytania}
	{
		\begin{list}
			{ \color{BurntOrange} Zadanie \arabic{ListCounter}:~}
			{
				\usecounter{ListCounter}
				\color{black}
			}
		}
		{
		\end{list}
	}
	\newcounter{mycounter}
	\newenvironment{Odpowiedz}
	{
		\begin{list}
			{ \color{Red} Odpowiedź  \Roman{mycounter}:~}
			{
				\usecounter{mycounter}
				\color{black}
			}
		}
		{
		\end{list}
	}
	\newcounter{bcounter}
	\newenvironment{Komentarz}
	{
		\begin{list}
			{ \color{RoyalPurple} {\#} Komentarz {\#}{\scshape}}
		}
		{
		\end{list}
	}
	
	\newenvironment{POdpowiedz}
	{
		\begin{list}
			{ \color{Green} {\*} Poprawna odpowiedź {\*}{\bfseries}}
		}  
		{
		\end{list}
	}
	\begin{Komentarz}
		\item Zanim przejdziemy do konkretnych instrukcji, warto wspomnieć, czym jest komentarz w kodzie. Jest to wydzielony fragment kodu, który nie jest w żaden sposób interpretowany przez interpreter języka Python. Ma on służyć jedynie człowiekowi, aby kod był czytelniejszy. Zazwyczaj jest to skomentowanie mniej oczywistych partii kodu, informacja o tym, do czego służy dana funkcja, czy po prostu wyłączenie wiersza kodu, którego chwilowo nie chcemy wykonywać.
	\end{Komentarz}
	\begin{Pytania}
		\item Poniżej przedstawiono schematy czterech kodów Python A,B,C oraz D,  w których jedną z funkcji jest pętla
		\renewcommand{\labelenumi}{\Alph{enumi}}
		\begin{enumerate}
			\item print
			\item input
			\item while
			\item round
		\end{enumerate}
		
		Spośród kodów oznaczonych literami A, B, C i D wybierz te, które pasują do wartości cyklu
		\begin{Odpowiedz}
			\item A
			\item B
			\item C
			\item D
		\end{Odpowiedz}
		\begin{POdpowiedz}
			\item III  \newline
			\newline
		\end{POdpowiedz}
		\item Gdzie zapisana jest zmienna "complex".
		\begin{Odpowiedz}
			\item 2+6j
			\item 1+4
		\end{Odpowiedz}
		\begin{POdpowiedz}
			\item Tylko I
		\end{POdpowiedz}
	\end{Pytania}
	\noindent Dokumenty Google:
	\begin{easylist}\newline
		\newline
		Otwórz plik w aplikacji Dokumenty Google.
		# Kliknij Edytuj Edytuj.
		# Kliknij ekran w miejscu, w którym chcesz dodać listę.
		# U góry ekranu kliknij Formatuj Formatowanie.
		# Kliknij Akapit.
		# Kliknij typ listy:
		## Lista numerowana Lista numerowana
		## Lista punktowana Lista punktowana
		# Aby zmienić styl listy, kliknij strzałkę w prawo Opcje listy.
		# Opcjonalnie: aby zmienić wcięcie listy, u dołu ekranu kliknij:
		## Zwiększ wcięcie Zwiększ wcięcie
		## Zmniejsz wcięcie Zmniejsz wcięcie
	\end{easylist}
	\begin{easylist}
		\ListProperties(Start1=1)
		# AAAAAAAAAAA
		## BBBBBBBBBB
		### CCCCCCCCCC
		#### DDDDDDDDDDDD
		##### FFFFFFFFFFFFFFFFFFFFF
		##### GGGGGGGGGGGGGGGGGGG
		####### HHHHHHHHHHHHHHHHHHHHHHH
		######## IIIIIIIIIIIIIIIIIIIIIIIIII
		######## JJJJJJJJJJJJJJJJJJJJJJJJJJJJJ
		########## JJJJJJJJJJJJJJJJJJJJJJJJJJJJJJJJ
		# AAAAAAAAAAA
		## BBBBBBBBBBBBBB
		### CCCCCCCCCCCC
		#### DDDDDDDDDDDD
	\end{easylist}
\end{document}