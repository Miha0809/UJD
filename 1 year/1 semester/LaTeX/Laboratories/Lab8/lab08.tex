\documentclass[mathserif]{beamer}
\usepackage{graphics}
\usepackage[table]{color}
\setbeamercolor{titlelike}{fg=cyan}
\setbeamercolor{background canvas}{bg=white}
\setbeamercolor{frametitle}{fg=white}
\setbeamercolor{normal text}{bg=black,fg=black}
\useoutertheme{infolines}
\usetheme{Antibes}
\title{Prezentacja na temat: LaTeX}
\author{Mykhailo Hulii}
\institute{Jana Długosza}
\date{\today}
\begin{document}
\frame{\titlepage}
    \begin{frame}
        \frametitle{LaTeX: co to jest?}
        LaTeX – oprogramowanie do zautomatyzowanego składu tekstu, a także związany z nim język znaczników, służący do formatowania dokumentów tekstowych i tekstowo-graficznych (na przykład: broszur, artykułów, książek, plakatów, prezentacji, a nawet stron HTML). Jego logo stylizowane jest z użyciem samego LaTeX-a jako {\displaystyle \mathrm {L\!\!^{{}_{\scriptstyle A}}\!\!\!\!\!\;\;T\!_{\displaystyle E}\!X} }.

LaTeX nie jest samodzielnym środowiskiem programistycznym: jest to zestaw makr stanowiących nadbudowę dla systemu składu TEX, automatyzujących czynności związane z procesem składania tekstu. Jednak, ze względu na dużą popularność LaTeX-a (w porównaniu z czystym TeX-em) nazwy te bywają używane zamiennie.

Twórcą pierwszej wersji LaTeX-a był Leslie Lamport, a powstała ona w laboratorium badawczym firmy SRI International. Pierwowzorem był język Scribe.
    \end{frame}
\begin{frame}{Informacja}
    \begin{scriptsize}
    Poprawna wymowa nazwy to latech lub ewentualnie lejtech. Zgermanizowana forma „lejtek” jest niepoprawna. Wymowa wynika ze źródłosłowu – ostatnia litera to greckie chi, jako że nazwa TeX wywodzi się z greckiego słowa τεχνη, oznaczającego umiejętność, sztukę, technikę.
    \end{scriptsize}
\vspace{0.5cm}
\begin{center}
\begin{tabular}{|p{2cm}|p{4cm}|}
\hline
Autor & Leslie Lamport\\
\hline
Pierwsze wydanie & 1984\\
\hline
Aktualna wersja stabilna & November 2022 LaTeX release\\
\hline
Rodzaj & skład tekstu\\
\hline
Licencja & LaTeX Project Public\\
\hline
\end{tabular}
\end{center}
\end{frame}
\end{document}
