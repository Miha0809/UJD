\documentclass[a4paper,11pt]{article}
\usepackage[T1]{fontenc}
\usepackage[utf8]{inputenc}
\usepackage{polski}
\usepackage{fancyvrb}
\usepackage{listings}
\usepackage{algorithmic}
\usepackage{algorithm2e}
\usepackage{dialogue}
\usepackage{todonotes}
\usepackage{blindtext}
\usepackage{hyperref}
\hypersetup{colorlinks = true,linkcolor=red,citecolor=green,urlcolor=cyan}
\title{laboratorium10}
\author{Mykhailo}
\date{\today}

\begin{document}

\section{Zadanie}
A = \{21,10,20,22\}\\
B = \{8,9,7,20,4,21\}\\

\begin{equation*}
    \begin{array}{l}
        A \cup B \\
        A \cup A = A \\
    \end{array}
 \end{equation*}

 \begin{equation}
    A \times B     
 \end{equation}

\section{Zadanie}
\begin{verbatim}
 lim=int(input("Podaj limit:"))
 sum=0
 for i in range(0,lim):
 if i<=0:
 continue
 sum+=i
 if sum>=lim:
 break
 print(sum)
\end{verbatim}
\section{Zadanie}
\begin{verbatim}
 def F(L):
 L.sort(reverse=True)
 return L
 def main():
 L=[1,2,3]
 print(F(L))
 main()
\end{verbatim}
\section{Zadanie}
\begin{Verbatim*}[frame=single]
 def Wypisz(a):
 print(a,':', type(a))
 def main():
 Wypisz(4)
 Wypisz("Hello")
 Wypisz(2.5)
 Wypisz(5+4j)
 main()
\end{Verbatim*}
\section{Zadanie}

\begin{lstlisting}[language=Python]
def main():
 N=10
 print("Tabliczka mnożenia")
 for w in  range(N):
 for k in range(N):
 print((w+1)*(k+1),end="")
 print()
 main()
\end{lstlisting}
\section{Zadanie}
\begin{algorithm}
$i = 10$\;
\eIf{$i\geq 5$}
{$i = i-1$\;}{\If{$i\leq 3$}
{$i = i+2$\;}}
\end{algorithm}
\section{Zadanie}
Python(\href{https://www.google.com/search?channel=fs&client=ubuntu&q=python}{link1}) – język programowania wysokiego poziomu ogólnego przeznaczenia, o rozbudowanym pakiecie bibliotek standardowych, którego ideą przewodnią jest czytelność i klarowność kodu źródłowego. Jego składnia cechuje się przejrzystością i zwięzłością.
Morzesz pobrać Python tu: \href{https://www.python.org/}{link2}
\section{Zadanie}
\begin{dialogue}
  \speak{Kasia} Uwielbiam gotować, to moja największa pasja.
\speak{Janek} Doprawdy?
\speak{Kasia}Tak, dlatego uczę się w szkole gastronomicznej.
\speak{Janek} Chcesz zostać w przyszłości kucharką?
\speak{Kasia} To moje marzenie.Chciałabym kiedyś otworzyć własną restaurację.
\speak{Janek} Wspaniały plan!Jakie dania chciałabyś serwować?
\speak{Kasia}Zdecydowanie przysmaki kuchni włoskiej.
\speak{Janek}Brzmi pysznie! Jestem pewien, że za kilka lat będziesz serwować najlepszą pizzę w mieście.  
\end{dialogue}
\section{Zadanie}
 \blindtext 
\todo{popraw tu} 
\end{document}
