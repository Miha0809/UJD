\documentclass[a4paper,12pt]{article}
\usepackage{graphicx}
\usepackage[T1]{fontenc}
\usepackage[utf8]{inputenc}
\usepackage{polski}
\usepackage{amsmath}
\usepackage{setspace}
\usepackage{amssymb}
\usepackage{mathtools}
\usepackage[left=1.5cm,right=1.5cm,top=2cm,bottom=2cm]{geometry}

\title{Laboratorium 12}
\author{Mykhailo Hulii}
\date{\today}

\begin{document}
	
	
	\maketitle 
	
	\section{Kod generujący poniższy tekst}
	
	Funkcją kwadratową nazywamy funkcję postaci $f(x)=ax^2+bx+c$, gdzie $a \neq 0$. Funkcją kwadratową nazywamy funkcję postaci:
	$$f(x)=ax^2+bx+c$$
	gdzie $a \neq 0$.
	Funkcją kwadratową nazywamy funkcję postaci
	$$f(x)=ax^2+bx+c$$
	gdzie $a \neq 0$.
	
	\section{Kod generujący poniższy tekst}
	$$A \cup B = \{\, x \colon (x \in A) \vee (x \in B)\,\}$$
	
	$$\sum_{k=1}^{\infty}\frac{1}{k^2+1}$$
	
	$$\lim_{n \to \infty} a_{n} = g \Leftrightarrow
	\forall \varepsilon > 0 \; \exists N_{\varepsilon
		\in \mathbb{N}} \; \forall n > N_{\varepsilon} \colon
	\left| {a_{n} - g} \right| < \varepsilon $$
	
	$$\int_{c}^{d} \left[ \int_{u(y)}^{v(y)} f(x,y)dx \right] dy$$
	
	\section{Kod generujący poniższy tekst}
	LATEX inaczej składa wzory w trybie matematycznym i tekstowym!$\lim_{n \to \infty} \sum_{k = 1}^n \frac{1}{k^2} =
	\frac{\pi^2}{6}$
	$$\lim_{n \to \infty} \sum_{k = 1}^n \frac{1}{k^2} =
	\frac{\pi^2}{6}$$
	
	$$\sqrt{2}\sqrt{x^2 + \sqrt{1 + \sqrt{\sqrt{2}-1}}}$$
	
	$$\frac{1}{1 + \frac{1}{1 + \frac{1}{1 +
				\frac{1}{1 + \dots}}}}$$
	
	
	\section{Kod generujący poniższy tekst}
	
	$f\colon X \to Y$\\ $f : X \to Y$
	
	
	
	
	\section{Kod generujący poniższy tekst}
	
	Ograniczniki to symbole nawiasów i inne podobne symbole, które mogą rozszerzać się pionowo.
	(x, y), {x, y}, [x, y]
	$(x,y)$, $\{x,y\}$, $[x,y]$, $\langle x,y \rangle$,
	$\lceil x,y \rceil$, $\lfloor x,y \rfloor$
	
	$$
	\left(\sum_{i = 0}^{10} i^2 \right)\qquad
	\left\{\sum_{i = 0}^{10} i^2 \right\}\qquad
	\left[\sum_{i = 0}^{10} i^2 \right\uparrow\qquad
	\left\vert\sum_{i = 0}^{10} i^2 \right.\qquad
	\left.\sum_{i = 0}^{10} i^2 \right\Vert
	$$
	
	
	
	\section{Kod generujący poniższy tekst}
	Środowisko array funkcjonuje podobnie jak tabular, ale służy do tworzenia
	struktur tabelarycznych zawierających wyrażenia matematyczne.
	$$
	U = \left( \begin{array}{cc} A & b \end{array} \right) =
	\left(
	\begin{array}{lllll}
		a_{11} & a_{12} & \dots & a_{1n} & b_1\\
		a_{21} & a_{22} & \dots & a_{2n} & b_2\\
		\vdots & \vdots & \ddots & \vdots & \vdots\\
		a_{m1} & a_{m2} & \dots & a_{mn} & b_m\\
	\end{array}
	\right)
	$$
	
	
	
	
	\section{Kod generujący poniższy kod}
	\begin{equation}
		\label{eq:funkcjaf}
		f(x) = \left\lbrace
		\begin{array}{rcl}
			-x^2 & \text{dla} & x \leqslant 0,\\
			\sqrt{x} + \sin x & \text{dla} & x > 0.
		\end{array}\right.
	\end{equation}
	
	\begin{equation}
		\left\{\begin{array}{l}
			(t_1^{+} - t_1^{-}) \circ C_{P’} = 0\\
			(t_2^{+} - t_2^{-}) \circ C_{P’} = 0\\
			\dots\\
			(t_m^{+} - t_m^{-}) \circ C_{P’} = 0
		\end{array}\right.
	\end{equation}
	
	\section{Kod generujący poniższy tekst}
\begin{align}
		(\sin x)’ & = \cos x, & (\cos x)’ & = -\sin x\\
		(\sin x)’’ & = -\sin x, & (\cos x)’’’ & = \sin x \\
\end{align}
	\begin{align}
		& \left|z\right| = 0 \iff z = 0,\label{eq:comp1}\\
		& \left|z\right| \geqslant 0,\label{eq:comp2}\\
		& \left|\frac{z_1}{z_2}\right| =
		\frac{\left|z_1\right|}{\left|z_2\right|},\label{eq:comp3}
	\end{align}
	\begin{align*}
		\sqrt{ax^2 + bx + c} &= \pm x \sqrt{a} \pm t, && a > 0\\
		\sqrt{ax^2 + bx + c} &= tx \pm \sqrt{c}, && c > 0\\
		\sqrt{ax^2 + bx + c} &= (x-x_1)t, && \Delta > 0.
	\end{align*}
	\begin{align*}
		\text{\bf Act}\colon &
		\frac{}{a.E \stackrel{a}{\longrightarrow} E} &
		\text{\bf Com}_3\colon &
		\frac{E \stackrel{a}{\longrightarrow} E’ \;\;
			F \stackrel{\bar{a}}{\longrightarrow} F’}
		{E|F \stackrel{\tau}{\longrightarrow} E’|F’}
		\\
		\text{\bf Sum}_{j}\colon &
		\frac{E_j \stackrel{a}{\longrightarrow} E_j’}
		{\sum_{i\in I} E_i \stackrel{a}{\longrightarrow} E_j’},
		\text{ gdzie }j \in I &
		\text{\bf Res}\colon &
		\frac{E \stackrel{a}{\longrightarrow} E’}
		{E\backslash L \stackrel{a}{\longrightarrow} E’\backslash L},
		\text{ gdzie }a,\bar{a}\notin L
	\end{align*}
	
	\begin{align*}
		x \in (U \cup V) \cap W
		& \iff (x \in U \cup V) \land x \in W,\\
		& \iff (x \in U \lor x \in V) \land x \in W,\\
		& \iff (x \in U \land x \in W) \lor (x \in V \land x \in W),\\
		\intertext{co wynika z zastosowania tożsamożsci ...,}
		& \iff (x \in U \cap W) \lor (x \in V \cap W),\\
		& \iff x \in (U \cap W) \cup (V \cap W).
	\end{align*}
	
	
	
	\section{Kod generujący poniższy tekst}
	Zapoznaj się z poleceniami: overline, overbrace, underline, underbrace,
	substack.
	
	\begin{align*}
		& \overline{m + n} \underline{x + y}\\
		& \underbrace{a + a + \ldots + a}_{n} = n \cdot a\\
		& A = \{\,\overbrace{n \in \mathbb{Z}}^{opis}\colon
		\overbrace{n \ne 0 \land n \mod 2 = 0}^{warunek}\,\}\\
		& k = \prod_{\substack{0 < i \leqslant n \\
				0 \leqslant j \leqslant n \\ i \ne j}}(i - j)
	\end{align*}
\end{document}
