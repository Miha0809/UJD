\documentclass[a4paper,12pt]{article}
\usepackage[T1]{fontenc}
\usepackage[utf8]{inputenc}
\usepackage{polski}
\usepackage{enumitem}
\usepackage{amssymb}

\begin{document}
	\begin{itemize}
		\item Pierwszy poziom item
		\item Pierwszy poziom item
		\begin{itemize}
			\item Drugi poziom item
			\item Drugi poziom item
			\begin{itemize}
				\item Trzeci poziom item
				\item Trzeci poziom item
				\begin{itemize}
					\item Czwarty poziom item
					\item Czwarty poziom item
				\end{itemize}
			\end{itemize}
		\end{itemize}
	\end{itemize}
	
	
	\renewcommand{\labelitemi}{$\blacksquare$}
	\renewcommand{\labelitemii}{$\square$}
	\begin{itemize}
		\item Pierwszy poziom item
		\item Pierwszy poziom item
		\begin{itemize}
			\item Drugi poziom item
			\item Drugi poziom item
			\begin{itemize}
				\item Trzeci poziom item
				\item Trzeci poziom item
				\begin{itemize}
					\item Czwarty poziom item
					\item Czwarty poziom item
				\end{itemize}
			\end{itemize}
		\end{itemize}
	\end{itemize}
	
	
	\begin{enumerate}
		\item Pierwszy poziom item.
		\item Pierwszy poziom item.
		\begin{enumerate}
			\item Drugi poziom item.
			\item Drugi poziom item.
			\begin{enumerate}
				\item Trzeci poziom item.
				\item Trzeci poziom item.
				\begin{enumerate}
					\item Czwarty poziom item.
					\item Czwarty poziom item.
				\end{enumerate}
			\end{enumerate}
		\end{enumerate}
	\end{enumerate}
	
	
	\begin{enumerate}
		\item Pierwszy poziom item
		\item Pierwszy poziom item
		\begin{itemize}
			\item Drugi poziom item
			\item Drugi poziom item
			\begin{enumerate}
				\item Trzeci poziom item
				\item Trzeci poziom item
				\begin{enumerate}
					\item Czwarty poziom item
					\item Czwarty poziom item
				\end{enumerate}
			\end{enumerate}
		\end{itemize}
	\end{enumerate}
	
	
	\begin{enumerate}
		\item Pierwszy poziom item
		\item Pierwszy poziom item
		\begin{itemize}
			\item Drugi poziom item
			\item Drugi poziom item
			\begin{enumerate}
				\item Trzeci poziom item
				\item Trzeci poziom item
				\begin{itemize}
					\textendash Czwarty poziom item\newline
					\textendash Czwarty poziom item
				\end{itemize}
			\end{enumerate}
		\end{itemize}
	\end{enumerate}
	
	
	\begin{enumerate}
		\item Pierwszy poziom item
		\item Pierwszy poziom item
		\begin{enumerate}
			\item Drugi poziom item
			\item Drugi poziom item
			\begin{enumerate}
				\item Trzeci poziom item
				\item Trzeci poziom item
				\begin{enumerate}
					\item Czwarty poziom item
					\item Czwarty poziom item
				\end{enumerate}
			\end{enumerate}
		\end{enumerate}
	\end{enumerate}
	
	
	\renewcommand{\labelenumi}{\Roman{enumi}}
	\renewcommand{\labelenumii}{ \Roman{enumi}.\arabic{enumii} }
	\renewcommand{\labelenumiii}{\Alph{enumiii}}
	\renewcommand{\labelenumiv}{\alph{enumiv}}
	\begin{enumerate}
		\item Pierwszy poziom item
		\item Pierwszy poziom item
		\begin{enumerate}
			\item Drugi poziom item
			\item Drugi poziom item
			\begin{enumerate}
				\item Trzeci poziom item
				\item Trzeci poziom item
				\begin{enumerate}
					\item Czwarty poziom item
					\item Czwarty poziom item
				\end{enumerate}
			\end{enumerate}
		\end{enumerate}
	\end{enumerate}
	
	
	\begin{itemize}
		\item Wartość domyślna dla 1 item.
		\item Wartość domyślna dla 2 item.
		\item[$\square$] Wartość specjalna nadana przez użytkownika.
	\end{itemize}
	
	
	\begin{itemize}[label= \textasteriskcentered]
		\item Pierwszy poziom item nienumerowanego.
		\item Pierwszy poziom item nienumerowanego.
	\end{itemize}
	\begin{enumerate}[label={(\alph*)}]
		\item Pierwszy poziom item numerowanego
		\item Pierwszy poziom item numerowanego
	\end{enumerate}
	
	
	\setitemize[1]{label=\textasteriskcentered}
	\setitemize[2]{label={--}}
	\setitemize[3]{label={\#}}
	\setitemize[4]{label={@}}
	\begin{itemize}
		\item Pierwszy poziom item
		\item Pierwszy poziom item
		\begin{itemize}
			\item Drugi poziom item
			\item Drugi poziom item
			\begin{itemize}
				\item Trzeci poziom item
				\item Trzeci poziom item
				\begin{itemize}
					\item Czwarty poziom item
					\item Czwarty poziom item
				\end{itemize}
			\end{itemize}
		\end{itemize}
	\end{itemize}
	
	
	\setenumerate[1]{label={\Alph*.}}
	\setenumerate[2]{label={(\alph*)}}
	\setenumerate[3]{label={(\Roman*)}}
	\setenumerate[4]{label={(\roman*)}}
	\begin{enumerate}
		\item Pierwszy poziom item
		\item Pierwszy poziom item
		\begin{enumerate}
			\item Drugi poziom item
			\item Drugi poziom item
			\begin{enumerate}
				\item Trzeci poziom item
				\item Trzeci poziom item
				\begin{enumerate}
					\item Czwarty poziom item
					\item Czwarty poziom item
				\end{enumerate}
			\end{enumerate}
		\end{enumerate}
	\end{enumerate}
	
	
	\begin{description}
		\item [ok. 4000 p.n.e.] - koło,
		\item [1740] \hfill \\ Najstarszy znany rysunek parowego działa samobieżnego,
		\item [1769] - artyleryjski ciągnik parowy Nicolasa Cugnota,
		\item [1801] - parowy trójkołowiec Richarda Trevithicka,
		\item [1811] - angielski wynalazca John Blenkinsop wraz
		Matthew Murrayem opatentował konstrukcję parowozu
		z kołem zębatym poruszającym się po zębatej szynie
		biegnącej z boku torów,
		\item [1825] - dyliżans parowy (omnibus) Gurneya w Anglii,
		\item [1827] - amerykański pojazd parowy Oshkosh Shomera i Farranda
		\item [1834] - dyliżans parowy Dietza we Francji,
		\item [1865] - angielska ustawa Ustawa o czerwonej fladze praktycznie
		zakazuje używania drogowych pojazdów parowych,
		\item [1875] - pierwszy pojazd z silnikiem spalinowym - Siegfried Marcus, Wiedeń.
	\end{description}
	
	
	\begin{enumerate}[label={(\arabic*)}]
		\setcounter{enumi}{4}
		\item Pierwszy poziom item
		\item Pierwszy poziom item
		\begin{enumerate}
			\setcounter{enumii}{4}
			\item Drugi poziom item
			\item Drugi poziom item
			\begin{enumerate}[label={\roman*.}]
				\setcounter{enumiii}{4}
				\item Trzeci poziom item
				\item Trzeci poziom item
				\begin{enumerate}[label={\Alph*. }]
					\setcounter{enumiv}{4}
					\item Czwarty poziom item
					\item Czwarty poziom item
				\end{enumerate}
			\end{enumerate}
		\end{enumerate}
	\end{enumerate}
	
\end{document}

