\documentclass[12pt,b5paper]{article}
\usepackage[T1]{fontenc}
\usepackage[utf8]{inputenc}
\usepackage{amsmath}
\usepackage{array}
\usepackage{polski}
\title{laboratorium 09}
\author{Mykhailo Hulii}
\date{08.12.2022}
\begin{document}
	\begin{center}
		\section{zadanie }
		Funkcją kwadratową nazywamy funkcję, którą można opisać wzorem:
		$f(x) = ax^2+bx+c$, gdzie $a \ne 0$
		Liczby rzeczywiste a, b oraz c nazywamy współczynnikami funkcji kwadratowej. Dziedziną funkcji kwadratowej jest zbiór liczb rzeczywistych.
		\section{zadanie}
		Z teorii względności wynika między innymi, że materia i energia są w pewnym sensie tym samym, a ich zależność od siebie opisuje wzór 
		\begin{equation} \label{eqn}
			E = {mc^2}
		\end{equation} w którym E oznacza energię, m - masę, a c - prędkość światła
	\section{zadanie}
		\begin{equation}\label{eq1}
			\begin{split}
				A & = \frac{\pi r^2}{2} \\ & = \frac{1}{2} \pi r^2
			\end{split}
		\end{equation}
	\section{zadanie}
	 Równaniem wykładniczym nazywamy równanie, w którym niewiadoma występuje w wykładniku potęgi. Sposób rozwiązania równania wykładniczego zależy od jego typu. 
			\begin{multline*}
			p(x) = x^8+x^7+x^6+x^5\\- x^4 - z^3 - x^2 - x
		\end{multline*}
	\section{zadanie}
	Wzory skróconego mnożenia pozwalają szybciej wykonywać obliczenia.
	Oto najczęściej stosowane wzory:
	Przykłady stosowania wszystkich powyższych wzorów znajdziesz kolejnych w podrozdziałach.
		\begin{align*}
			a+b & a-b & (a+b)(a-b)\\x+y & x-y & (x+y)(x-y)\\ p+q & p-q & (p+q)(p-q)
		\end{align*}
	\section{zadanie}
		\[\left \{\begin{tabular}{ccc}
			1 & 5 & 8 \\ 0 & 2 & 4 \\ 3 & 3 & -8
		\end{tabular}\right \} \]
	\section{zadanie}
	Tożsamości trygonometryczne	W tym ustępie podajemy około 30 tożsamości trygonometrycznych, których biegłe opanowanie pamięciowe, wraz z dowodami, uważamy za niezbędne.  
		\[\sin^2(a)+\cos^2(a)=1\]
		\section{zadanie}
		\[\lim_{h \rightarrow 0 }\frac{f(x+h)-f(x)}{h}\]
		Ten operator zmienia się , gdy jest używany obok tekstem \(\lim_{x \rightarrow h} \) 
		\section{zadanie}
		\[\binom{n}{k} = \frac{n!}{k!(n-k)!}\]
		\section{zadanie}
		$$
		U = \left(
		\begin{array}{cc} A & b \end{array} \right) =
		\left(\begin{array}{lllll}
			a_{11} & a_{12} & \dots & a_{1n} & b_1 \\
			a_{21} & a_{22} & \dots & a_{2n} & b_2 \\
			\vdots & \vdots & \ddots & \vdots & \vdots \\
			a_{m1} & a_{m2} & \dots & a_{mn} & b_m \\
		\end{array}
		\right)
		$$
		\section{zadanie}
		$$
		\lambda(2^{\alpha}) = 2^{\beta - 2}
			\begin{cases}
			\beta = \alpha, & \text{dla } \alpha \geq 3 \\ \beta = 3, & \text{dla }\alpha = 2\\
			\beta = 2, & \text{dla }\alpha = 1 
			\end{cases} $$  
		\section{zadanie}
		
		\begin{subequations}
			\label{eq:resExamle}
			\begin{align}
				A & = a.b.c.A \label{eq:rese1} \\
				B & = A \backslash \{c\} \label{eq:rese2}
			\end{align}
		\end{subequations}
		
	\end{center} 
\end{document}
