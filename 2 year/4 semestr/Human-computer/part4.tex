\begin{center}
  \section{Część 4 - Mykhailo Hulii}
  \subsection{Projekt koncepcyjny rozwiązań interakcji}
\end{center}

\subsubsection{Problemy do rozwiązania}

\textbf{Tabela 1: Interakcje ogólne}
\begin{table}[h]
  \begin{tabular}{|p{1cm}|p{\dimexpr\textwidth-1cm\relax}|} \hline
    \textbf{Nr} & \textbf{Problem} \\\hline
    1 & Użytkownik uruchamia aplikację \\\hline
    2 & Aplikacja wyświetla ekran powitalny \\\hline
    3 & Użytkownik wykonuje akcje: naciśnięcie przycisku, wpisuje: login, hasło i powtórzenie hasła, wpisywanie tekstu z klawiatury, zaznaczanie elementów graficznych, używanie przycisków, przełączników i ikon do wykonywania akcji, klikanie, przeciąganie, przesuwanie elementów interaktywnych \\\hline
    4 & Aplikacja reaguje na akcje użytkownika i dostarcza informacji zwrotnych \\\hline
    5 & Użytkownik wykonuje zadania, rozwiązuje problemy lub udziela odpowiedzi \\\hline
  \end{tabular}
\end{table}

\textbf{Tabela 2: Komunikacja tekstowa}
\begin{table}[h]
  \begin{tabular}{|p{1cm}|p{\dimexpr\textwidth-1cm\relax}|} \hline
    \textbf{Nr} & \textbf{Problem} \\\hline
    1 & Pacjent przychodzi do przychodni, rejestruje się do lekarza, po rozmowie z lekarzem, lekarz zapisuje wszystkie dane w aplikacji (IKP) \\\hline
    2 & Lekarz poszukuje lekki w aplikacji i mówi przybliżoną cenę za leczenie i dodaje recepty w IKP \\\hline
    3 & Użytkownik naciska na lekarza, przechodzi do wiadomości prywatnych i pisze lekarzu jeżeli ma jeszcze pytanie \\\hline
    4 & Użytkownik wybiera symptomy na dziś i wskaże jaką dawkę leku przyjął \\\hline
    5 & Użytkownik ma mdostęm do receptów i w aptece może go pokazać \\\hline
  \end{tabular}
\end{table}

\textbf{Tabela 3: Nawigacja pacjenta}
\begin{table}[h]
  \begin{tabular}{|p{1cm}|p{\dimexpr\textwidth-1cm\relax}|} \hline
    \textbf{Nr} & \textbf{Problem} \\\hline
    1 & Czat z lekarzem \\\hline
    2 & Symptomy, gdzie pacjent zaznacza swoje symptomy oraz wskaże jaką dawkę leku przyjął \\\hline
    3 & Wszystkie aktualne recepty, który wypisał lekarz \\\hline
    4 & Proil użytkownika \\\hline
  \end{tabular}
\end{table}

\newpage

\textbf{Tabela 4: Nawigacja lekarza}
\begin{table}[h]
  \begin{tabular}{|p{1cm}|p{\dimexpr\textwidth-1cm\relax}|} \hline
    \textbf{Nr} & \textbf{Problem} \\\hline
    1 & Wszystkie czaty z pacjentami oraz lista pacjentów \\\hline
    2 & Lista wszystkich leków \\\hline
    3 & Lista szablonów receptów \\\hline
    4 & Profil użytkownika \\\hline
  \end{tabular}
\end{table}
