\begin{center}
  \section{Część 7 - Krystsina Shemshenia}
  \subsection{Plan i realizacja testów użyteczności z badaniem ankietowym (kwestionariusz oceny prototypu)}
\end{center}

\quad Stworzenie darmowej aplikacji zdrowotnej, która jest prosta w obsłudze, pryjazna dla oka, zapisuje calość inforacji, zaczynają od tabletek do stanu użytkownika i udostępniania ją dla lekarza. I to że ona będę dostępna na wszystkich urządzeniach z Androidem, IOS, a także korzystanie z przeglądarki przyniesie wiele korzyści dla użytkownika.

\begin{center}
  \subsection{Planowanie testów użyteczności}
\end{center}

\paragraph{Definiowanie celów testów:} 
\begin{itemize}
  \item Konto użytkownika.
  \item Możliwość zmiany danych użytkownika.
  \item Możliwość personalizacji aplikacji pod użytkownika.
  \item Możliwość dodawania danych zdrowotnych.
  \item Możliwość dodawanie urządzenia z którymi aplikacja pracuje.
\end{itemize}

\paragraph{Grupa docelowa:} Testy użyteczności zostaną przeprowadzone na grupie użytkowników w wieku od 18 do 65 lat, którzy korzystają z aplikacji mobilnych do zarządzania zdrowiem i aktywnością fizyczną. Grupa będzie obejmować osoby z różnym poziomem zaawansowania technicznego.

\paragraph{Scenariusze testowe:} Scenariusze testowe zostaną opracowane na podstawie najczęściej wykonywanych czynności w aplikacji:
\begin{itemize}
  \item Rejestracja i logowanie.
  \item Dodawanie danych zdrowotnych.
  \item Monitorowanie aktywności fizycznej.
  \item Ustawianie przypomnień o lekach.
  \item Przeglądanie historii zdrowia.
\end{itemize}

\paragraph{Metody badawcze:} 
\begin{itemize}
  \item \textbf{Testy moderowane:} Użytkownicy będą wykonywać zadania w obecności moderatora, który będzie obserwował ich interakcje z aplikacją i zbierał informacje na temat problemów napotkanych przez użytkowników.
  \item \textbf{Testy niemoderowane:} Użytkownicy będą samodzielnie korzystać z aplikacji, a ich działania będą rejestrowane za pomocą narzędzi do analizy użyteczności.
\end{itemize}

\begin{center}
  \subsection{Realizacja testów użyteczności}
\end{center}

\paragraph{Przygotowanie testów:}
\begin{itemize}
  \item Przygotowanie prototypu aplikacji do testów.
  \item Opracowanie szczegółowych instrukcji dla uczestników testów.
  \item Przygotowanie kwestionariuszy oceny prototypu.
\end{itemize}

\paragraph{Przeprowadzenie testów:}
\begin{itemize}
  \item Rekrutacja uczestników testów.
  \item Przeprowadzenie testów moderowanych i niemoderowanych.
  \item Zbieranie danych jakościowych i ilościowych z testów.
\end{itemize}

\paragraph{Analiza wyników:}
\begin{itemize}
  \item Analiza danych zebranych podczas testów.
  \item Identyfikacja głównych problemów z użytecznością.
  \item Opracowanie rekomendacji dotyczących ulepszeń.
\end{itemize}

\begin{center}
  \subsection{Kwestionariusz oceny prototypu}
\end{center}

\paragraph{Analiza wyników:}
\begin{itemize}
  \item Jak oceniasz ogólną łatwość obsługi aplikacji?
    \begin{itemize}
      \item Bardzo łatwa
      \item Łatwa
      \item Średnia
      \item Trudna
      \item Bardzo trudna
    \end{itemize}
  \item Jak oceniasz nawigację w aplikacji?
    \begin{itemize}
      \item Bardzo intuicyjna
      \item Intuicyjna
      \item Średnia
      \item Mało intuicyjna
      \item Bardzo nieintuicyjna
    \end{itemize}
  \item Jak oceniasz wygląd i estetykę aplikacji?
    \begin{itemize}
      \item Bardzo atrakcyjna
      \item Atrakcyjna
      \item Średnia
      \item Mało atrakcyjna
      \item Bardzo nieatrakcyjna
    \end{itemize}
\end{itemize}

\paragraph{Pytania szczegółowe dotyczące funkcji:}
\begin{itemize}
  \item Czy proces rejestracji był dla Ciebie łatwy do wykonania?
    \begin{itemize}
      \item Tak
      \item Częściowo
      \item Nie
    \end{itemize}
  \item Jak oceniasz funkcję monitorowania aktywności fizycznej?
    \begin{itemize}
      \item Bardzo dobra
      \item Dobra
      \item Średnia
      \item Słaba
      \item Bardzo słaba
    \end{itemize}
  \item Czy przypomnienia o lekach są dla Ciebie pomocne?
    \begin{itemize}
      \item Tak
      \item Nie
    \end{itemize}
  \item Czy napotkałeś/aś problemy techniczne podczas korzystania z aplikacji?
    \begin{itemize}
      \item Tak
      \item Nie
    \end{itemize}
\end{itemize}

\paragraph{Pytania otwarte:}
\begin{itemize}
  \item Co najbardziej Ci się podobało w aplikacji?
  \item Jakie elementy aplikacji sprawiły Ci największe trudności?
  \item Jakie zmiany chciałbyś/chciałabyś zobaczyć w przyszłych wersjach aplikacji?
\end{itemize}
