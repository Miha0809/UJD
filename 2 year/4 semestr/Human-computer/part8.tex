\begin{center}
  \section{Część 8 - Krystsina Shemshenia}
  \subsection{Opracowanie raportu z testów użyteczności}
\end{center}

\quad Testy użyteczności aplikacji zostały przeprowadzone w celu oceny intuicyjności, funkcjonalności oraz ogólnego doświadczenia użytkownika. Niniejszy raport zawiera wyniki przeprowadzonych testów, zidentyfikowane problemy oraz rekomendacje dotyczące usprawnień, które mogą zwiększyć satysfakcję użytkowników i efektywność korzystania z aplikacji.

\begin{center}
  \subsection{Przeprowadzenie testów}
\end{center}

\paragraph{Metodyka:} Testy użyteczności zostały przeprowadzone z wykorzystaniem dwóch głównych metod: testów moderowanych oraz niemoderowanych. Testy moderowane pozwoliły na bieżąco obserwować zachowania użytkowników i uzyskiwać bezpośrednie informacje zwrotne, natomiast testy niemoderowane umożliwiły zarejestrowanie naturalnych interakcji użytkowników z aplikacją bez wpływu zewnętrznych czynników.

\paragraph{Uczestnicy:} Testy przeprowadzono na grupie 20 użytkowników w wieku od 18 do 65 lat, reprezentujących różne poziomy zaawansowania technicznego oraz zróżnicowane potrzeby zdrowotne. Wśród uczestników znalazło się 10 kobiet i 10 mężczyzn. Grupa została dobrana w taki sposób, aby odzwierciedlać różnorodność przyszłych użytkowników aplikacji.

\paragraph{Narzędzia:} Do przeprowadzenia testów użyto następujących narzędzi:
\begin{itemize}
  \item Prototyp aplikacji w wersji mobilnej.
  \item Formularze oceny użyteczności.
  \item Narzędzia do rejestrowania interakcji użytkowników (np. nagrywanie ekranu, notatki obserwatora).
  \item Ankiety przed i po teście w celu zebrania danych demograficznych oraz opinii użytkowników.
\end{itemize}

\begin{center}
  \subsection{Scenariusze testowe}
\end{center}

\paragraph{Scenariusz 1} Rejestracja nowego użytkownika:
\begin{itemize}
  \item Użytkownik uruchamia aplikację.
  \item Użytkownik wybiera opcję "Rejestracja".
  \item Użytkownik wybiera metodę rejestracji (np. Google, Facebook, Apple ID, e-mail).
  \item Użytkownik wypełnia formularz rejestracyjny (podaje e-mail, hasło, potwierdza hasło).
  \item Użytkownik akceptuje regulamin i politykę prywatności.
  \item Użytkownik klika przycisk "Zarejestruj się".
  \item Użytkownik otrzymuje e-mail z linkiem aktywacyjnym i aktywuje konto.
\end{itemize}

\paragraph{Scenariusz 2} Logowanie istniejącego użytkownika:
\begin{itemize}
  \item Użytkownik uruchamia aplikację.
  \item Użytkownik wybiera opcję "Logowanie".
  \item Użytkownik wybiera metodę logowania (np. Google, Facebook, Apple ID, e-mail).
  \item Użytkownik wprowadza e-mail i hasło.
  \item Użytkownik klika przycisk "Zaloguj się".
  \item Użytkownik zostaje przekierowany na stronę główną aplikacji.
\end{itemize}

\paragraph{Scenariusz 3} Dodawanie danych zdrowotnych:
\begin{itemize}
  \item Użytkownik loguje się do aplikacji.
  \item Użytkownik przechodzi do sekcji "Zdrowie".
  \item Użytkownik wybiera opcję "Dodaj dane zdrowotne".
  \item Użytkownik wprowadza dane dotyczące parametrów zdrowotnych (np. ciśnienie krwi, tętno, poziom glukozy we krwi).
  \item Użytkownik zapisuje wprowadzone dane.
\end{itemize}

\paragraph{Scenariusz 4} Monitorowanie aktywności fizycznej:
\begin{itemize}
  \item Użytkownik loguje się do aplikacji.
  \item Użytkownik przechodzi do sekcji "Aktywność".
  \item Użytkownik synchronizuje aplikację z urządzeniem noszonym.
  \item Użytkownik sprawdza statystyki dotyczące aktywności fizycznej (np. liczba kroków, przebyta odległość, spalone kalorie).
  \item Użytkownik ustawia cele dotyczące aktywności fizycznej.
\end{itemize}

\paragraph{Scenariusz 5} Ustawianie przypomnień o lekach:
\begin{itemize}
  \item Użytkownik loguje się do aplikacji.
  \item Użytkownik przechodzi do sekcji "Leki".
  \item Użytkownik wybiera opcję "Dodaj lek".
  \item Użytkownik wprowadza nazwę leku, dawkę oraz harmonogram przyjmowania.
  \item Użytkownik ustawia przypomnienia na określone godziny.
  \item Użytkownik zapisuje ustawienia przypomnień.
\end{itemize}

\paragraph{Scenariusz 6} Przeglądanie historii zdrowia:
\begin{itemize}
  \item Użytkownik loguje się do aplikacji.
  \item Użytkownik przechodzi do sekcji "Historia zdrowia".
  \item Użytkownik wybiera zakres dat, które chce przeglądać.
  \item Użytkownik przegląda historię swoich parametrów zdrowotnych oraz wizyt lekarskich.
  \item Użytkownik korzysta z opcji filtrowania i sortowania danych.
\end{itemize}

\begin{center}
  \subsection{Wyniki testów}
\end{center}

\paragraph{Ogólne wrażenia użytkowników:} Większość uczestników oceniła aplikację jako intuicyjną i estetyczną. Podkreślano, że interfejs użytkownika jest przejrzysty, a nawigacja w aplikacji nie sprawia trudności. Poniżej przedstawiono szczegółowe wyniki oceny poszczególnych funkcji aplikacji.

\subsubsection{Zidentyfikowane problemy i rekomendacje}

\paragraph{Rejestracja i logowanie:}
\begin{itemize}
  \item \textbf{Problem:} Niektórzy użytkownicy mieli trudności z wyborem metody rejestracji, nie wiedząc, którą z dostępnych opcji wybrać.
  \item \textbf{Rekomendacja:} Dodanie krótkich opisów przy każdej metodzie rejestracji, które wyjaśnią, w jaki sposób każda metoda działa i jakie są jej korzyści.
\end{itemize}

\paragraph{Dodawanie danych zdrowotnych:}
\begin{itemize}
  \item \textbf{Problem:} Użytkownicy mieli trudności ze znalezieniem sekcji odpowiedzialnej za dodawanie danych zdrowotnych.
  \item \textbf{Rekomendacja:} Wyraźniejsze oznaczenie tej sekcji oraz dodanie przewodnika krok po kroku, który pomoże użytkownikom zrozumieć proces dodawania danych.
\end{itemize}

\paragraph{Monitorowanie aktywności fizycznej:}
\begin{itemize}
  \item \textbf{Problem:} Brak opcji ręcznego wprowadzania danych dla użytkowników, którzy nie korzystają z urządzeń zewnętrznych.
  \item \textbf{Rekomendacja:} Dodanie możliwości ręcznego wprowadzania danych dotyczących aktywności fizycznej, co pozwoli na bardziej kompleksowe monitorowanie zdrowia.
\end{itemize}

\paragraph{Ustawianie przypomnień o lekach:}
\begin{itemize}
  \item \textbf{Problem:} Część użytkowników miała trudności z ustawieniem dokładnych godzin przypomnień, co prowadziło do pomyłek.
  \item \textbf{Rekomendacja:} Uproszczenie procesu ustawiania godzin poprzez dodanie predefiniowanych opcji oraz wizualnego interfejsu, który ułatwi wybór godzin.
\end{itemize}

\paragraph{Przeglądanie historii zdrowia:}
\begin{itemize}
  \item \textbf{Problem:} Brak możliwości filtrowania i sortowania danych w historii zdrowia sprawiał, że użytkownicy mieli trudności ze znalezieniem konkretnych informacji.
  \item \textbf{Rekomendacja:} Dodanie opcji filtrowania i sortowania danych według daty, typu aktywności oraz innych kryteriów, co ułatwi użytkownikom zarządzanie swoimi danymi zdrowotnymi.
\end{itemize}

\subsubsection{Szczegółowa analiza ankiet}

\quad Wyniki ankiet oceny prototypu dostarczyły szczegółowych informacji na temat doświadczeń użytkowników:

\paragraph{Łatwość obsługi:}
\begin{itemize}
  \item Bardzo łatwa: 8 użytkowników
  \item Łatwa: 7 użytkowników
  \item Średnia: 5 użytkowników
  \item Trudna: 0 użytkowników
  \item Bardzo trudna: 0 użytkowników
\end{itemize}

\paragraph{Nawigacja:}
\begin{itemize}
  \item Bardzo intuicyjna: 7 użytkowników
  \item Intuicyjna: 10 użytkowników
  \item Średnia: 3 użytkowników
  \item Mało intuicyjna: 0 użytkowników
  \item Bardzo nieintuicyjna: 0 użytkowników
\end{itemize}

\paragraph{Estetyka aplikacji:}
\begin{itemize}
  \item Bardzo atrakcyjna: 9 użytkowników
  \item Atrakcyjna: 8 użytkowników
  \item Średnia: 3 użytkowników
  \item Mało atrakcyjna: 0 użytkowników
  \item Bardzo nieatrakcyjna: 0 użytkowników
\end{itemize}

\paragraph{Ogólna satysfakcja:}
\begin{itemize}
  \item Bardzo zadowolony: 10 użytkowników
  \item Zadowolony: 7 użytkowników
  \item Neutralny: 3 użytkowników
  \item Niezadowolony: 0 użytkowników
  \item Bardzo niezadowolony: 0 użytkowników
\end{itemize}

\begin{center}
  \subsection{Wnioski i rekomendacje}
\end{center}

\paragraph{Ogólna satysfakcja:} Testy użyteczności wykazały, że aplikacja jest ogólnie dobrze oceniana przez użytkowników, którzy uważają ją za intuicyjną i estetyczną. Niemniej jednak zidentyfikowano kilka obszarów wymagających poprawy, aby zwiększyć użyteczność i satysfakcję użytkowników.

\paragraph{Szczegółowe rekomendacje:}
\begin{itemize}
  \item \textbf{Rejestracja i logowanie:} Dodanie krótkich opisów przy metodach rejestracji.
  \item \textbf{Dodawanie danych zdrowotnych:} Wyraźniejsze oznaczenie sekcji oraz przewodnik krok po kroku.
  \item \textbf{Monitorowanie aktywności fizycznej:} Dodanie możliwości ręcznego wprowadzania danych.
  \item \textbf{Ustawianie przypomnień o lekach:} Uproszczenie procesu ustawiania godzin przypomnień.
  \item \textbf{Przeglądanie historii zdrowia:} Dodanie opcji filtrowania i sortowania danych.
\end{itemize}

\begin{center}
  \subsection{Podsumowanie raportu}
\end{center}

\quad Przeprowadzone testy użyteczności dostarczyły cennych informacji, które pomogą w dalszym doskonaleniu aplikacji. Wdrożenie rekomendacji powinno przyczynić się do poprawy ogólnego doświadczenia użytkownika oraz zwiększenia satysfakcji z korzystania z aplikacji. Dalsze testy powinny być przeprowadzane cyklicznie, aby na bieżąco monitorować efektywność wprowadzanych zmian oraz dostosowywać aplikację do zmieniających się potrzeb użytkowników.
