\begin{center}
  \section{Część 6 - Anna Kulesha}
  \subsection{Budowa prototypu interaktywneg szczegółowego (ekranowego)}
\end{center}

\subsubsection{Prezentacja informacji użytkownikowi w systemie AidMedical}

\paragraph{Nacisk zostanie położony na:}
\begin{itemize}
  \item \textbf{Strukturę ekranów:}
    \begin{itemize}
      \item Określenie układu różnych sekcji informacji, takich jak nagłówki, treść, nawigacja, przyciski, etc., aby zapewnić przejrzystość i łatwość nawigacji.
      \item Wyraźny podział między sekcjami pacjenta i lekarza, aby każda grupa użytkowników miała łatwy dostęp do niezbędnych funkcji.
    \end{itemize}
  \item \textbf{Porządek i hierarchia:}
    \begin{itemize}
      \item Zastosowanie odpowiedniego układu, aby wyróżnić najważniejsze informacje, takie jak dane pacjentów, wskaźniki zdrowotne, recepty, etc., co pomoże użytkownikom w szybkim znalezieniu potrzebnych danych.
      \item Użycie nagłówków, podziałów sekcji i odpowiednich stylów tekstu, aby stworzyć czytelną hierarchię informacji.
    \end{itemize}
  \item \textbf{Spójność stylistyczna:}
    \begin{itemize}
      \item Utrzymanie jednolitego stylu graficznego na wszystkich ekranach, aby zapewnić spójne doświadczenie użytkownikom.
      \item Stosowanie spójnej kolorystyki, ikonografii oraz typografii.
    \end{itemize}
\end{itemize}

\quad System będzie korzystał z odpowiednich ikon, symboli i kolorów, aby wyróżnić różne rodzaje informacji i ułatwić ich zrozumienie. Ikony będą używane do identyfikacji różnych typów danych, takich jak wskaźniki zdrowotne, leki, czy informacje kontaktowe.

\subsubsection{Przykład ekranów w systemie AidMedical}

\begin{itemize}
  \item \textbf{Ekran logowania:}
    \begin{itemize}
      \item Pola do wprowadzenia loginu i hasła z opcją „zapomniałem hasła”.
      \item Przycisk „Zaloguj się” wyraźnie widoczny i łatwo dostępny.
    \end{itemize}
  \item \textbf{Ekran główny pacjenta:}
    \begin{itemize}
      \item Sekcja powitalna z informacjami osobistymi i danymi kontaktowymi.
      \item Wskaźniki zdrowotne, takie jak ciśnienie, poziom cukru, etc., przedstawione w czytelny sposób.
      \item Lista leków i przypomnienia o ich zażywaniu.
      \item Sekcja do zapisywania codziennych objawów.
    \end{itemize}
  \item \textbf{Ekran główny lekarza:}
    \begin{itemize}
      \item Lista pacjentów wraz z możliwością wyszukiwania i sortowania.
      \item Dostęp do szczegółowych danych pacjentów, takich jak diagnozy, leczenie, choroby przewlekłe, etc.
      \item Możliwość aktualizacji danych pacjentów oraz przeglądania tygodniowych raportów.
    \end{itemize}
\end{itemize}

\quad Każda sekcja systemu będzie zaprojektowana z myślą o maksymalnej użyteczności i łatwości dostępu do informacji, z uwzględnieniem specyficznych potrzeb zarówno pacjentów, jak i lekarzy.

% \includepdf[scale=0.925,pages=-]{AidMedical.pdf}
\includepdf[pages=-]{AidMedical.pdf}

